%!TeX root=../tese.tex
%("dica" para o editor de texto: este arquivo é parte de um documento maior)
% para saber mais: https://tex.stackexchange.com/q/78101/183146

% As palavras-chave são obrigatórias, em português e em inglês, e devem ser
% definidas antes do resumo/abstract. Acrescente quantas forem necessárias.
\keyword{Software Complexity}
\keyword{Software Metrics}
\keyword{Machine Learning Systems}
\keyword{Machine Learning \break Engineering}
\keyword{MLOps}

% O resumo é obrigatório, em português e inglês. Estes comandos também
% geram automaticamente a referência para o próprio documento, conforme
% as normas sugeridas da USP.
\abstract{
The Software Crisis has reached AI: according to Gartner's report in 2021, only
around 53\% of AI products successfully reach production. Since the early days of
software engineering, the rise of complexity has been known to play a key role
in projects failing. The goal of this research is to investigate how
complexity affects Machine Learning (ML). On the one hand,
ML-enabled systems are essentially complex: they require multiple components to
train and serve ML models, and evolve along three axes of change -- code, model,
and data. On the other hand, handling this complexity is very much a firefighting
activity: there is no established organizational culture, processes, or
practices. In response to this, a critical research question emerges: how can
the complexity of ML-enabled systems be managed effectively? To address this
question, this research aims to introduce a metrics-based architectural model to
characterize the complexity of an ML-enabled system. Methodologically, the
metrics would support architectural decisions, providing a guideline for the
inception and evolution of these systems. Moreover, the characterization would
offer a test-bed to elaborate better organizational strategies for the future.
To understand the state of the art regarding ML-enabled systems, this proposal
relies on the literature of the nascent field of Machine Learning Engineering
(MLE). Although the literature is rich on software complexity metrics, fewer
results explore how the data and, in particular, the models affect the overall
complexity of ML-enabled systems. As a consequence, this research aims to
pioneer this discussion. To create the characterization model, it proposes two
case studies based on real ML-enabled systems. The first will be based on the
SPIRA project, developed at the University of São Paulo (USP) in Brazil. The
SPIRA system was architected in 2020 and developed by multiple undergrad
students supervised as part of this research. At an early stage, the goal is to
rely on SPIRA's open-source code and open-access documentation -- including
published and ongoing results -- as a sandbox to mature the characterization
model. The second case study will be done with industry partners from the
Jheronimus Academy of Data Science (JADS) in the Netherlands. JADS has a rich
ecosystem that bridges academia and industry, and has been hosting this research
since 2023. At a later stage, the goal is to empirically validate the
applicability and usefulness of the characterization model on industry systems.
By using effective metrics to track complexity, we hope new ML-enabled systems
have a higher probability of reaching production.
}
