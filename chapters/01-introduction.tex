%!TeX root=../thesis.tex
%("dica" para o editor de texto: este arquivo é parte de um documento maior)
% para saber mais: https://tex.stackexchange.com/q/78101/183146

\chapter{Introduction}
\label{chap:introduction}
%%%%%%%%%%%%%%%%%%%%%%%%%%%%%%%%%%%%%%%%%%%%%%%%%%%%%%%%%%%%%%%%%%%%%%%%%%%%%%%%

  % data science is cool
  % data science hype => ml spread
  % ml-based systems are complex
  % complexity has 2 parts: essential, accidental
  % ml-based systems have higher essential complexity
  % ml-based systems can have higher accidental complexity
  % most ml-based systems are not delivered
  % failture comes from various reasons: experience, teams, infrastructure
  % those are reflected into the architecture, becoming accidental complexity
  % measuring complexity in ml-enabled systems => finding signals of failure
  % finding signals of failture => improve the delivery of ml-based systems

  % *Why* ML-enabled systems are complex?
  Since 2011, the interest in the term \emph{Data Science} has grown steadily:
  ``data scientist'' has been considered ``the sexiest job of the 21st century''%
  ~\parencite{Cao2018DataScience,Davenport2012DataCentury}. Big tech companies
  such as Microsoft, Google, and Apple have been steadily delivering more
  AI-based products. This experience soon led their software engineers to
  an important understanding: models are only a small part of real-world
  ML-based systems~\parencite{Hulten2018BuildingSystems,Sculley2015HiddenSystems}.
  In fact, ML-based applications evolve via three different axes of change:
  code, model, and data~\parencite{Sato2019ContinuousLearning}.
  This makes them inherently more complex than traditional software-based
  information systems~\parencite{Amershi2019SoftwareStudy,
  Burkov2020MachineEngineering}.

  % *How* ML-enabled systems are complex?
  \emph{Complexity} has been a subject of discussion since the early
  days of the Software Engineering field~\parencite{Brooks1975TheMan-Month,
  Parnas1985TheSystems}. In the book~\citetitle{Brooks1975TheMan-Month},
  \citeauthor{Brooks1975TheMan-Month} introduces the concept of
  \emph{essential} and \emph{accidental} complexity for software%
  ~\parencite{Brooks1975TheMan-Month}:
    the \emph{essential} exists intrinsically to satisfy the requirements
    of the problem solved, whereas the \emph{accidental} may originate
    from any external factors that influence the solution chosen.
  Under this definition, an ML-enabled system has high essential
  complexity: it requires extra components in its architecture to
  support data processing and model handling%
  ~\parencite{Amershi2019SoftwareStudy,Lakshmanan2020MachinePatterns}.
  % As a consequence, this also creates more opportunities for
  % introducing accidental complexity into the system.

  % *Why* complexity matters?
  According to Gartner's reports~\parencite{Gartner2020,Gartner2022},
  only around 53-54\% of AI projects successfully reach production.
  Besides technical challenges~\parencite{Sculley2015HiddenSystems, 
  Thung2012AnSystems}, many factors influence the ability to deliver, such as
    developer experience~\parencite{MartinCleanCode2008,Reilly2022TheChange},
    team composition~\parencite{Nahar2021MoreProjects,Skelton2019TeamFlow},
    development processes ~\parencite{Nahar2021MoreProjects,
    Shankar2022OperationalizingStudy, Wazir2023MLOps:Review}, and
    infrastructure~\parencite{Davis2019CloudPatterns}.
  All of them end up reflected into the software architecture%
  ~\parencite{Brooks1975TheMan-Month, Ford2021SoftwareParts,
  Richards2020FundamentalsApproach}, thus becoming potential
  sources of accidental complexity.

  This proposal presents a plan to create a \emph{metrics-oriented
  architectural model to characterize the complexity of ML-enabled
  systems}. By understanding how complexity emerges in the software
  architecture, this research aims to provide a method to
  identify some of the pitfalls that make ML-enabled systems
  fail to reach production.

  % -- developer experience, team composition, infrastructure, etc.

  % Under this definition, creating an ML-enabled system where a traditional
  % software-based one suffices is its own category of accidental complexity
  % -- a practice strongly discouraged by introductory texts to the field%
  % ~\parencite{Burkov2020MachineEngineering,Hulten2018BuildingSystems,
  % Wilson2022MachineAction}.
  % All can become a source of accidental complexity in a system. 

  % However, when an ML-enabled system is adequate, handling its higher complexity
  % may be even more important to build, deliver, and maintain successful products.
  % As a consequence,
  % building ML-enabled systems requires considering a whole AI hierarchy of needs%
  % ~\parencite{Rogati2017TheNeeds}, whereas maintaining them demands rigor
  % against many types of technical debt~\parencite{Sculley2015HiddenSystems}.
  % Unfortunately, Data Science alone does not focus on these requirements%
  % ~\parencite{Burkov2020MachineEngineering,Makinen2021WhoHelp,Menzies2020TheAI,
  % Sato2019ContinuousLearning}.

  % \section{Problem Outline}
  % \label{sec:problem_outline}
  %%%%%%%%%%%%%%%%%%%%%%%%%%%%%%%%%%%%%%%%%%%%%%%%%%%%%%%%%%%%%%%%%%%%%%%%%%%%%%

  \section{Research Questions}
  \label{sec:research_questions}
  %%%%%%%%%%%%%%%%%%%%%%%%%%%%%%%%%%%%%%%%%%%%%%%%%%%%%%%%%%%%%%%%%%%%%%%%%%%%%%

    To achieve the goal of this research, this proposal introduces three
    main research questions and two sub-questions. They were defined
    according to the SMART principles, i.e., they should be Specific,
    Measurable, Achievable, Relevant, and Time-Bound.

    \begin{researchquestion}
      What are the complexity metrics related to the architecture of
      ML-enabled systems?
    \end{researchquestion}

    The first research question aims to understand which complexity
    metrics are available in the literature. \emph{Software Metrics}
    have already been thoroughly explored by the software engineering field%
    ~\parencite{Fenton2014SoftwareEdition}. However, ML-enabled systems also
    have their data and model dependencies~\parencite{Sato2005RNAFields},
    which fundamentally influence the functionality of the system.
    Finding metrics that describe the complexity of data and model
    related decisions \emph{beyond code} is the biggest potential
    challenge for answering this research question.

    \begin{researchquestion}
      How can complexity metrics be measured over the architecture of
      ML-enabled systems?
    \end{researchquestion}

    The second research question aims to create a process to collect
    the metrics found in \cref{rq:1}.
    % ML-enabled systems have their own
    % intricacies, which emerge as different design and architectural patterns
    % ~\parencite{Lakshmanan2020MachinePatterns,Washizaki2019StudyingSystems}.
    Metrics have different levels of abstraction: they may be related
    to code snippets, components, services, or dependencies of a system
    \parencite{Fenton2014SoftwareEdition}. Measuring a metric may require
    different levels of access to the system (e.g., having the codebase
    available for processing) or rely on different representations of
    the system (e.g., creating a graph describing its data flow).
    As a consequence, only some metrics found in \cref{rq:1} may
    be practical to collect. In particular, if two metrics provide
    similar information, it may be preferable to use the simpler.
    The subset of metrics worth collecting will become the
    \emph{metrics-oriented architectural model} this research
    proposes to create. Choosing which metrics should be included
    is the main challenge for answering this research question.
    
    % \begin{subresearchquestion}
    %   How should an ML-enabled system be prepared for collecting
    %   complexity metrics?
    % \end{subresearchquestion}
    
    % \begin{subresearchquestion}
    %   Which subset of complexity metrics should be chosen to characterize
    %   the architecture of ML-enabled systems?
    % \end{subresearchquestion}

    \begin{researchquestion}
      How can complexity metrics be used to aid the development of
      real-world ML-enabled systems?
    \end{researchquestion}

    The third research question aims to understand the usefulness and
    potential impact of the \emph{metrics-oriented architectural model}
    resulting from \cref{rq:2}. If the model succeeds in characterizing
    the complexity, it should aid engineers building ML-enabled systems.
    Since this can happen in multiple ways, this research will focus on
    two development tasks, described via the following sub-questions.
    
    \begin{subresearchquestion}
      How can complexity metrics be used to choose between architecture
      proposals for an ML-enabled system?
    \end{subresearchquestion}

    This sub-question aims to understand if the \emph{metrics-oriented
    architectural model} proposed in \cref{rq:2} can help discussions
    about how to \emph{evolve} ML-enabled systems. During the lifecycle of a
    system, such discussions are usually motivated by new major functional
    requirements. The goal is to \emph{avoid introducing accidental complexity},
    since the more complex the system, the harder it is to understand
    and maintain. For such purpose, metrics can be particularly useful
    to objectively measure the impact of changes. The greatest challenge
    for this sub-question is to explain the \emph{metrics-oriented
    architectural model} for developers, and then test if it improves
    their ability to choose between architecture proposals.

    \begin{subresearchquestion}
      How can complexity metrics be used to identify refactoring
      opportunities in an ML-enabled systems?
    \end{subresearchquestion}

    Similarly, this sub-question aims to understand if the
    \emph{metrics-oriented architectural model} proposed in \cref{rq:2}
    can help discussions about how to \emph{improve} ML-enabled systems.
    During the lifecycle of a system, such discussions are usually motivated
    by paying technical debt. The goal is to \emph{reduce accidental complexity},
    since the more complex the system, the harder it is to understand and
    maintain. For such purpose, metrics can be particularly useful to
    objectively measure the impact of changes. The greatest challenge
    for this sub-question is to explain the \emph{metrics-oriented
    architectural model} for developers, and then test if it improves
    their ability to identify refactoring opportunities.

  \section{Proposal Structure}
  \label{sec:proposal_structure}
  %%%%%%%%%%%%%%%%%%%%%%%%%%%%%%%%%%%%%%%%%%%%%%%%%%%%%%%%%%%%%%%%%%%%%%%%%%%%%%

    The remaining of this proposal is structured as follows.
      \Cref{chap:machine_learning_engineering} introduces the field of
      Machine Learning Engineering and formally discusses the definition
      of ML-enabled systems.
      \Cref{chap:measuring_complexity} provides a preliminary literature
      review about software metrics.
      \Cref{chap:research_methodology} delineates the research
      methodology, whose goal is to answer the research questions
      presented in \cref{sec:research_questions}.
      Finally, \Cref{chap:work_plan} presents the work plan to execute
      this research up to the end of this PhD.

% % [ ] What is the big data phenomenon?
% Data is being produced on a scale never seen in history. The so-called
% ``big data phenomenon'' has been represented by three changes in how
% data is collected and processed:
%   the volume, going from terabytes up to zettabytes of storage;
%   the variety, going from structured to more unstructured sources; and
%   the velocity, going from batch to streaming%
% ~\parencite{Sagiroglu2013BigReview}.
% Meanwhile, reasoning over and making decisions from large amounts of data
% has become a challenge and an opportunity for both academia and industry
% worldwide~\parencite{Lu2020ArtificialInnovation}.

% % [ ] What is Data Science?
% Data Science (DS) is an interdisciplinary field that mixes different skill
% sets to collect, organize, and analyze data \parencite{Hayashi1998WhatExample}.
% Traditionally, these tasks were in the realm of Statistics, which provides
% techniques to create insights from data. However, handling big data also
% requires knowledge about other fields, such as Computer Science, Business,
% and Sociology~\parencite{Cao2018DataScience,Dhar2013DataPrediction}.
% This enables creating data products that solve problems such as
% classification, recommendation, or decision-making%
% ~\parencite{Cao2018DataScience,Lakshmanan2020MachinePatterns}.

% % [ ] What are Artificial Intelligence and Machine Learning?
% There is no generally accepted definition for Artificial Intelligence (AI)%
% ~\parencite{Emmert-Streib2020ArtificialStatus}. As a field, the goal of AI
% is to understand and build intelligent entities, encompassing methods
% from different areas~\parencite{RussellS2021Artificial4th}. Among them,
% Machine Learning (ML) comprehends techniques that can learn from data,
% producing empirical solutions for problems~\parencite{Abu-Mostafa2012LearningData}.
% These broad descriptions blurry the line between the domains of Statistics,
% AI, and ML~\parencite{Emmert-Streib2020ArtificialStatus}. Regardless,
% the task of modeling, i.e., creating models from data, has become
% one of the primary responsibilities of modern data scientists%
% ~\mbox{\parencite{Burkov2020MachineEngineering,Kim2016TheTeams,
% Lakshmanan2020MachinePatterns}}.

% % What are the limitations of Data Science?
% Since 2011, the interest in the term Data Science has grown steadily,
% with ``data scientist'' considered ``the sexiest job of the 21st century''%
% ~\parencite{Cao2018DataScience,Davenport2012DataCentury}. Big tech companies
% such as Google, Microsoft, and Apple started developing and launching more
% AI-based products. Their experience led to an important understanding:
% models make only a small part of real-world intelligent systems%
% ~\mbox{\parencite{Hulten2018BuildingSystems,Sculley2015HiddenSystems}}.
% Furthermore, AI/ML applications have three different axes of change:
% code, model, and data~\parencite{Sato2019ContinuousLearning}.
% This makes them inherently more complex than traditional software-based
% information systems~\parencite{Amershi2019SoftwareStudy}. In fact, building
% intelligent systems requires considering a whole AI hierarchy of needs%
% ~\parencite{Rogati2017TheNeeds}, whereas maintaining them demands rigor
% against many types of technical debt~\parencite{Sculley2015HiddenSystems}.
% Unfortunately, Data Science alone does not focus on these requirements%
% ~\parencite{Burkov2020MachineEngineering,Makinen2021WhoHelp,Menzies2020TheAI,
% Sato2019ContinuousLearning}.

% \clearpage

% % [ ] What is Machine Learning Engineering?
% Machine Learning Engineering (MLE) combines scientific principles, tools,
% and techniques from both Machine Learning and Software Engineering to build
% intelligent systems, including all stages from data collection to modeling
% and delivery~\parencite{Burkov2020MachineEngineering}. MLE complements
% Data Science by addressing the remaining infrastructure required by AI/ML
% systems, encompassing their axes of change~\parencite{Ozkaya2020WhatSystems,
% Sato2019ContinuousLearning}. Likewise, \emph{machine learning engineers}
% are a new set of specialized software engineers that help data scientists
% to build, deliver, and operate intelligent systems%
% ~\mbox{\parencite{Burkov2020MachineEngineering,Lakshmanan2020MachinePatterns}}.

% % [ ] What is Intelligent Software Engineering? What is MLOps?
% The study and adaptation of Software Engineering practices for AI/ML
% systems have received different names in both academia and industry:
% \begin{itemize}
%   \item Intelligent Software Engineering -- ISE
%         \parencite{Xie2018IntelligentEngineering},
%   \item Software Engineering for Machine Learning -- SE4ML
%         \parencite{Amershi2019SoftwareStudy,
%                    Martinez-Fernandez2022SoftwareSurvey,
%                    Sculley2015HiddenSystems}, and
%   \item Software Engineering for Artificial Intelligence -- SE4AI 
%         \parencite{Carleton2020TheSE,
%                    Kastner2020TeachingSystems,
%                    Martinez-Fernandez2022SoftwareSurvey,
%                    Menzies2020TheAI}.
% \end{itemize}
% Although there is no consensus, the last term seems to be the most popular.
% Nevertheless, this project will use them interchangeably.

% % [ ] What is MLOps?
% Among SE4AI practices, Continuous Delivery for Machine Learning (CD4ML)
% has received particular attention~\parencite{Sato2019ContinuousLearning}.
% It extends the concept of Continuous Delivery (CD) to address the
% complexities of AI/ML systems. Meanwhile, Machine Learning Operations,
% also known as MLOps, became a trend in academia and industry%
% ~\parencite{Tamburri2020SustainableChallenges}. Inspired by the
% DevOps movement, it describes a new generation of infrastructure tools
% that help to manage different stages of the lifecycle of an intelligent system%
% ~\mbox{\parencite{Gift2021PracticalModels,Stenac2020IntroducingMLOps}}.
% In summary, MLOps enables CD4ML similarly to how DevOps enables CD.

% % [ ] Where are the limits of what we know?
% MLE, SE4AI, CD4ML, and MLOps are all relatively new concepts.
% % ~\parencite{Burkov2020MachineEngineering,Martinez-Fernandez2022SoftwareSurvey,
% %             Sato2019ContinuousLearning,Tamburri2020SustainableChallenges}.
% They were born from the challenges of delivering successful intelligent
% systems, and they grow motivated by the potential impact of new AI/ML
% products. Surveys evidence the growing interest in the subject, but also
% highlight important gaps in the literature: it is an area in its infancy%
% ~\parencite{Giray2021AChallenges, Martinez-Fernandez2022SoftwareSurvey, 
% Tamburri2020SustainableChallenges}.

% Given this context, this Ph.D. research is driven by the general research
% question: \emph{how can we reliably build, deliver and operate successful
% intelligent systems?} Defining ``successful'' is a challenge on its own.
% Nevertheless, this project will use the modern definition from the CHAOS
% Report made by the Standish Group \parencite{Wojewoda2015CHAOS2015}:
% the system was delivered on a reasonable estimated time, stayed on budget,
% and delivered satisfaction for its users regardless of its original scope.
% Therefore, to address this research question, the thesis of this Ph.D.
% research is that it is possible to \emph{document processes and patterns
% that allow the successful development of intelligent systems}.

% This research internship project is divided as follows.
%   \Cref{chap:fundamentals} further introduces concepts relevant
%   to this research.
%   \Cref{chap:objectives} presents the objectives for the research.
%   Lastly, \Cref{chap:research_plan} shows the research plan with
%   a timetable for its execution.