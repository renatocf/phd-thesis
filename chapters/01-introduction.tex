%!TeX root=../thesis.tex
%("dica" para o editor de texto: este arquivo é parte de um documento maior)
% para saber mais: https://tex.stackexchange.com/q/78101/183146

\chapter{Introduction}
\label{chap:introduction}
%%%%%%%%%%%%%%%%%%%%%%%%%%%%%%%%%%%%%%%%%%%%%%%%%%%%%%%%%%%%%%%%%%%%%%%%%%%%%%%%

% data science is cool
% data science hype => ml spread
% ml-based systems are complex
% complexity has 2 parts: essential, accidental
% ml-based systems have higher essential complexity
% ml-based systems can have higher accidental complexity
% most ml-based systems are not delivered
% failture comes from various reasons: experience, teams, infrastructure
% those are reflected into the architecture, becoming accidental complexity
% measuring complexity in ml-enabled systems => finding signals of failure
% finding signals of failture => improve the delivery of ml-based systems

% *Why* ML-enabled systems are complex?
Since 2011, the interest in the term \emph{Data Science} has grown steadily:
``data scientist'' has been considered ``the sexiest job of the 21st century''%
~\parencite{Cao2018DataScience,Davenport2012DataCentury}. Big tech companies
such as Microsoft, Google, and Apple have been steadily delivering more
AI-based products. This experience soon led their software engineers to
an important understanding: models are only a small part of real-world
ML-based systems~\parencite{Hulten2018BuildingEngineering,Sculley2015HiddenSystems}.
In fact, ML-based applications evolve via three different axes of change:
code, model, and data~\parencite{Alves2024PracticesReview,Sato2019ContinuousLearning}.
This makes them inherently more complex than traditional software-based
systems~\parencite{Diaz-De-Arcaya2023ASurvey, Giray2021AChallenges,
Granlund2021MLOpsCases}.
% systems~\parencite{Amershi2019SoftwareStudy, Belani2019RequirementsSystems,
% Burkov2020MachineEngineering, Diaz-De-Arcaya2023ASurvey, Giray2021AChallenges,
% Granlund2021MLOpsCases}.

% *How* ML-enabled systems are complex?
\emph{Complexity} has been a subject of discussion since the early
days of the Software Engineering field~\parencite{Brooks1975TheMan-Month,
Parnas1985TheSystems}. In the book~\citetitle{Brooks1975TheMan-Month},
\citeauthor{Brooks1975TheMan-Month} introduces the concept of
\emph{essential} and \emph{accidental} complexity for software%
~\parencite{Brooks1975TheMan-Month}:
  the \emph{essential} exists intrinsically to satisfy the requirements
  of the problem solved, whereas the \emph{accidental} may originate
  from any external factors that influence the solution chosen.
Under this definition, an ML-enabled system has high essential
complexity: it requires extra components in its architecture to
support data processing and model handling%
~\parencite{Ameisen2020BuildingApplications, Amershi2019SoftwareStudy,
Benton2020MachineApplications}.
%, Lakshmanan2020MachineMLOps, Huyen2022DesigningApplications, 
% Wilson2022MachineAction}.
% As a consequence, this also creates more opportunities for
% introducing accidental complexity into the system.

% *Why* complexity matters?
According to Gartner's reports~\parencite{Gartner2020,Gartner2022},
only around 54\% of AI projects successfully reach production.
Besides technical challenges~\parencite{Sculley2015HiddenSystems, 
Thung2012AnSystems}, many factors influence the ability to deliver, such as
  infrastructure~\parencite{Davis2019CloudPatterns,Morris2020InfrastructureCode},
  developer experience~\parencite{MartinCleanCode2008,Reilly2022TheChange},
  development processes~\parencite{Nahar2021MoreProjects,
  Shankar2022OperationalizingStudy, Wazir2023MLOps:Review}, and
  team composition~\parencite{Nahar2021MoreProjects,Skelton2019TeamFlow}.
All of them reflect into the software architecture%
~\parencite{Brooks1975TheMan-Month, Ford2021SoftwareParts,
Richards2020FundamentalsApproach}, thus becoming potential
sources of accidental complexity.

% This proposal presents a plan to create a \emph{metrics-oriented
% architectural model to characterize the complexity of ML-enabled
% systems}. By using metrics to identify where complexity emerges
% in the software architecture of ML-enabled systems, this research
% aims to provide a method to avoid pitfalls that make them fail
% to reach production.

This research plan outlines \emph{a metrics-oriented architectural
model to characterize the complexity of ML-enabled systems}.
The goal is to use metrics to identify where complexity
emerges in the software architecture of ML-enabled systems.
Hopefully, this technique will assist developers to manage
complexity, allowing them reach production more often.

% -- developer experience, team composition, infrastructure, etc.

% Under this definition, creating an ML-enabled system where a traditional
% software-based one suffices is its own category of accidental complexity
% -- a practice strongly discouraged by introductory texts to the field%
% ~\parencite{Burkov2020MachineEngineering,Hulten2018BuildingSystems,
% Wilson2022MachineAction}.
% All can become a source of accidental complexity in a system. 

% However, when an ML-enabled system is adequate, handling its higher complexity
% may be even more important to build, deliver, and maintain successful products.
% As a consequence,
% building ML-enabled systems requires considering a whole AI hierarchy of needs%
% ~\parencite{Rogati2017TheNeeds}, whereas maintaining them demands rigor
% against many types of technical debt~\parencite{Sculley2015HiddenSystems}.
% Unfortunately, Data Science alone does not focus on these requirements%
% ~\parencite{Burkov2020MachineEngineering,Makinen2021WhoHelp,Menzies2020TheAI,
% Sato2019ContinuousLearning}.

  \section{Problem Outline}
  \label{sec:problem_outline}
  %%%%%%%%%%%%%%%%%%%%%%%%%%%%%%%%%%%%%%%%%%%%%%%%%%%%%%%%%%%%%%%%%%%%%%%%%%%%%%

  According to~\citeauthor{Burkov2020MachineEngineering}, the field of
  \emph{Machine Learning Engineering} (MLE) combines scientific principles,
  tools, and techniques from both \emph{Machine Learning} (ML) and
  \emph{Software Engineering} (SWE) to build ML-enabled systems%
  ~\parencite{Burkov2020MachineEngineering},
  including all stages from data collection to modeling and delivery%
  ~\parencite{Amershi2019SoftwareStudy,Burkov2020MachineEngineering,
  Hilllaz2016TrialsStudy}.
  MLE complements \emph{Data Science} (DS) by addressing the surrounding
  infrastructure required by ML-enabled systems~\parencite{Nahar2021MoreProjects}.
  % encompassing their axes of change: code, model, and data%
  % ~\parencite{Alves2024PracticesReview,Sato2019ContinuousLearning}.
  Likewise, \emph{Machine Learning Engineers} (MLEs) are a new set of
  specialized developers that help data scientists to build, deliver,
  and operate ML-enabled systems~\parencite{Burkov2020MachineEngineering,
  Hilllaz2016TrialsStudy, Kreuzberger2023MachineArchitecture}.
  % Lakshmanan2020MachineMLOps, Wilson2022MachineAction}.

  The study of development practices and processes for ML-enabled systems
  has been receiving different names%
  \footnote{%
    \emph{Artificial Intelligence for Software Engineering} (AI4SE) and 
    synonyms describe the study of ML-enabled tools to assist development%
    ~\parencite{Kastner2020TeachingSystems,Martinez-Fernandez2022SoftwareSurvey},
    like when using \href{https://github.com/features/copilot}{GitHub Copilot}
    % for software development.
    They are not the focus of this research.
  }. This research will adopt the latter:
  %--------------------------------------------------------------------------%
  \begin{itemize}
    \item \emph{Intelligent Software Engineering} (ISE)
          \parencite{Andrade2021ContinuousStudy,
                     DeugoPatternsEngineering.,
                     Xie2018IntelligentEngineering},
    \item \emph{Software Engineering for Machine Learning} (SE4ML)
          \parencite{Amershi2019SoftwareStudy,
                     Lorenzoni2021MachineReview,
                     Nascimento2020SoftwareReview,
                     Martinez-Fernandez2022SoftwareSurvey,
                     Sculley2015HiddenSystems}, and
    \item \emph{Software Engineering for Artificial Intelligence} (SE4AI) 
          \parencite{Carleton2020TheSE,
                     Giray2021AChallenges,
                     Heck2020TurningEngineers,
                     Kastner2020TeachingSystems,
                     Martinez-Fernandez2022SoftwareSurvey,
                     Menzies2020TheAI}.
  \end{itemize}
  %--------------------------------------------------------------------------%
  % Since the last term seems to be the most popular, this research will adopt it.
  % Although there is no consensus, the last term seems to be the most popular
  % in recent literature. Therefore, this research will adopt it.

  % % [ ] What is MLOps?
  Among SE4AI practices, \emph{Continuous Delivery for Machine Learning} (CD4ML)
  has received particular attention~\parencite{Sato2019ContinuousLearning}.
  It reuses the concept of \emph{Continuous Delivery} (CD)%
  ~\parencite{Humble2010ContinuousAutomation}: applications should be developed
  in small safe increments, and always be in a releasable state~\parencite{Kim2021TheOrganizations}.
  However, it also extends it, considering ML-enabled systems can change because
  of code, model, and data~\parencite{Sato2019ContinuousLearning}.
  Meanwhile, the term \emph{Machine Learning Operations} (MLOps) became a trend
  in academia and industry~\parencite{Tamburri2020SustainableChallenges}.
  Inspired by \emph{DevOps}~\parencite{Forsgren2018Accelerate:Organizations,
  Kim2016TheTeams}, it describes a new generation of infrastructure tools
  to manage ML-enabled systems~\parencite{Gift2021PracticalModels,
  Mboweni2022ADevOps,Stenac2020IntroducingMLOps}.
  In summary, MLOps enables CD4ML similarly to how DevOps enables CD.

  MLE, SE4AI, CD4ML, and MLOps are all relatively new concepts%
  ~\parencite{Burkov2020MachineEngineering,Sato2019ContinuousLearning,
  Tamburri2020SustainableChallenges},
  whose relationships are an ongoing discussion%
  ~\parencite{Kreuzberger2023MachineArchitecture,Wazir2023MLOps:Review}.
  They were born from the challenges of delivering successful ML-enabled systems,
  and they grow motivated by the potential impact of new AI/ML products.
  The number of surveys published since 2022 evidence the interest in the area%
  ~\parencite{
    Diaz-De-Arcaya2023ASurvey,
    FaustinoAmorim2023AnSurvey,
    Foidl2024DataDevelopers,
    % Giray2021AChallenges,
    Kreuzberger2023MachineArchitecture,
    % Jahic2020StateSystems,
    % Lwakatare2020FromSystems,
    Martinez-Fernandez2022SoftwareSurvey,
    Mboweni2022ADevOps,
    Priestley2023APipelines,
    Paleyes2022ChallengesStudies,
    Steidl2023ThePractice,
    Wazir2023MLOps:Review,
  }:
  it is an area in its infancy.
  
  % Tamburri -> complexity is unsustainable!
  % Giraya -> we need to handle complexity!
  % Shankar -> practitioners want to avoid complexity! (they have their tricks for that)

  While each survey summarizes important gaps in the SE4AI literature,
  only \citeauthor{Giray2021AChallenges} highlights \emph{handling complexity}
  as an open challenge for MLE~\parencite{Giray2021AChallenges}. However, this
  idea is also supported by \citeauthor{Tamburri2020SustainableChallenges}, who
  outlines challenges for sustainable AI~\parencite{Tamburri2020SustainableChallenges}.
  \mbox{In an interview} study with 18 professional ML engineers%
  ~\parencite{Shankar2022OperationalizingStudy},
  \citeauthor{Shankar2022OperationalizingStudy} describe how participants
  \Quote{expressed an aversion to complexity}, implying practitioners
  consider it an issue.

  % Kreuzberger -> we need to coordinate components and infrastructure
  % Diaz-de-Arcaya -> hard to modularize
  % Hillaz -> discussing about complexity is not objective
  % -> research gap => metrics about complexity to find where it is

  The survey by \citeauthor{Diaz-De-Arcaya2023ASurvey} further
  explores possible origins of complexity in ML-enabled systems%
  ~\parencite{Diaz-De-Arcaya2023ASurvey}, all of which become
  visible in the software architecture:
  %--------------------------------------------------------------------------%
  \begin{itemize}
    \item they are harder to modularize,
          since data modeling and data processing are highly coupled%
          ~\parencite{Wan2021HowPractices};
    \item they require an understanding of ML principles and techniques, 
          with implications to the number of ML models required, their size,
          versioning and tracking~\parencite{Lwakatare2020Large-ScaleSolutions,
          Priestley2023APipelines}; and
    \item they often have to handle Big Data~\parencite{Sagiroglu2013BigReview},
          which leads them to adopt domain-specific distributed data
          processing patterns~\parencite{Amershi2019SoftwareStudy,
          Benton2020MachineApplications,Foidl2024DataDevelopers}.
  \end{itemize}
  %--------------------------------------------------------------------------%
  
  The field study carried out by~\citeauthor{Hilllaz2016TrialsStudy} discusses the
  impact of such complexity for practitioners~\parencite{Hilllaz2016TrialsStudy}.
  It was held with 11 participants with at least two years of experience
  in the ML field.
    Seasoned engineers described that developing ML-enabled systems
    \Quote{required skills held only by certain `high priests'},
    while debugging them was
    \Quote{something akin to magic and even voodoo}%
  ~\parencite{Hilllaz2016TrialsStudy}.
  Such challenges become even more critical since ML-enabled systems need
  to be updated and improved continuously~\parencite{Wan2021HowPractices}.
  % These challenges to understand and communicate about complexity increases
  % as ML-enabled systems need to be updated and improved continuously.

  In summary, there is an opportunity for research about the complexity
  of ML-enabled systems~\parencite{Giray2021AChallenges,
  Shankar2022OperationalizingStudy,Tamburri2020SustainableChallenges},
  which manifests directly in the software architecture%
  ~\parencite{Diaz-De-Arcaya2023ASurvey}, challenging the \emph{evolution}
  and \emph{maintainability} of these applications%
  ~\parencite{Hilllaz2016TrialsStudy,Wan2021HowPractices}.
  The many surveys in the literature show a general lack of studies
  on \emph{metrics} for ML-enabled systems. Henceforth, this research
  intends to explore this gap, thus understanding how metrics can help
  to manage the architectural complexity of ML-enabled systems.

  % The survey by \citeauthor{Kreuzberger2023MachineArchitecture} highlights
  % how there is much literature on the challenge of creating ML models,
  % but little on the problem of building and operating production-ready
  % ML-enabled systems~\parencite{Kreuzberger2023MachineArchitecture}.
  % \citeauthor{Diaz-De-Arcaya2023ASurvey} provides a possible explanation:
  % the complexity of ML-enabled systems requires a high degree of expertise
  % from developers.
  
  % building nor all components related to a ML-enabled system.
  
  \section{Research Questions}
  \label{sec:research_questions}
  %%%%%%%%%%%%%%%%%%%%%%%%%%%%%%%%%%%%%%%%%%%%%%%%%%%%%%%%%%%%%%%%%%%%%%%%%%%%%%

  To achieve the goal of this research, this proposal introduces three main
  research questions and two sub-questions. They follow the SMART principle%
  ~\parencite{Verschuren2010DesigningDesign}, i.e., they should be
  \emph{Specific, Measurable, Achievable, Relevant, and Time-Bound}.

  %--------------------------------------------------------------------------%
  \begin{researchquestion}
    What are the measurable dimensions of complexity in
    the architecture of ML-enabled systems?
  \end{researchquestion}
  %--------------------------------------------------------------------------%

  The first research question aims to explore metrics related to
  complexity available in the literature. \emph{Software Metrics}
  have been thoroughly researched by Software Engineering%
  ~\parencite{Fenton2014SoftwareEdition}.
  However, ML-enabled systems also have their data and model dimensions%
  ~\parencite{Alves2024PracticesReview,Sato2005RNAFields},
  which affect the software architecture.
  Finding metrics that measure the data- and model-related complexity 
  \emph{beyond code} is the biggest potential challenge for
  answering~\cref{rq:1}.

  %--------------------------------------------------------------------------%
  \begin{researchquestion}
    How can complexity metrics be operationalized over
    the architecture of ML-enabled systems?
  \end{researchquestion}
  %--------------------------------------------------------------------------%

  The second research question aims to create a process to collect
  the metrics found in \cref{rq:1}.
  % ML-enabled systems have their own
  % intricacies, which emerge as different design and architectural patterns
  % ~\parencite{Lakshmanan2020MachinePatterns,Washizaki2019StudyingSystems}.
  Metrics may be related to different abstraction levels of a system:
  from code snippets, to components, to services%
  ~\parencite{Fenton2014SoftwareEdition, Tu2009TheMetrics}.
  Moreover, they can have varying levels of different quality attributes%
  ~\parencite{Latva-Koivisto2001FindingModels, Polancic2017ComplexityReview},
  such as validity, reliability, computability, intuitiveness,
  ease of implementation, and independence of other metrics.
  As a consequence, only some metrics found in \cref{rq:1} may
  be practical or useful to collect. In particular, if two metrics
  provide similar information, it will be preferable to use the best
  according to the quality attributes. Regardless, all metrics should
  be collected over exemplary projects, which can represent the
  components of a common ML-enabled system. The subset of metrics
  selected, and the requirements and techniques to measure them,
  will compose the \emph{metrics-oriented architectural model}
  this research proposes to create. Therefore, the challenge
  for answering~\cref{rq:2} is twofold: finding exemplary projects
  to collect the metrics, and then choosing the subset of metrics.
  
  % Measuring % a metric may require different levels of access to the system
  % (e.g., having the codebase available for processing) or rely
  % on different representations of the system (e.g., creating a
  % graph describing its data flow).
  % As a consequence, only some metrics found in \cref{rq:1} may
  % be practical to collect. In particular, if two metrics provide
  % similar information, it may be preferable to use the simpler.
  
  %--------------------------------------------------------------------------%
  \begin{researchquestion}
    How can complexity metrics be used to aid the development,
    operation, and evolution of real-world ML-enabled systems?
  \end{researchquestion}
  %--------------------------------------------------------------------------%

  The third research question aims to understand the usefulness and
  potential impact of the \emph{metrics-oriented architectural model}
  resulting from \cref{rq:2}. If the model succeeds in characterizing
  the complexity, it should aid engineers building and maintaining
  ML-enabled systems. Since this can happen in multiple ways, this
  research will focus on two development tasks, described via the
  following sub-questions of~\cref{rq:3}.
  
  %--------------------------------------------------------------------------%
  \begin{subresearchquestion}
    How can complexity metrics be used to choose between architecture
    proposals for an ML-enabled system?
  \end{subresearchquestion}
  %--------------------------------------------------------------------------%

  This sub-question aims to understand if the \emph{metrics-oriented
  architectural model} proposed in \cref{rq:2} can help discussions
  about how to \emph{evolve} ML-enabled systems. During the lifecycle of a
  system, such discussions are usually motivated by new major functional
  requirements. The goal is to \emph{avoid introducing accidental complexity},
  since the more complex the system, the harder it is to understand
  and maintain. For such purpose, metrics can be particularly useful
  to objectively measure the impact of changes. The greatest challenge
  for~\cref{rq:3.1} is to explain the \emph{metrics-oriented
  architectural model} to developers, and then test if it improves
  their ability to choose between architecture proposals.

  %--------------------------------------------------------------------------%
  \begin{subresearchquestion}
    How can complexity metrics be used to identify refactoring
    opportunities in an ML-enabled systems?
  \end{subresearchquestion}
  %--------------------------------------------------------------------------%

  Similarly, this sub-question aims to understand if the
  \emph{metrics-oriented architectural model} proposed in \cref{rq:2}
  can help discussions about how to \emph{improve} ML-enabled systems.
  During the lifecycle of a system, such discussions are usually motivated
  by paying technical debt. The goal is to \emph{reduce accidental complexity},
  since the more complex the system, the harder it is to understand and
  maintain. For such purpose, metrics can be particularly useful to
  objectively measure the impact of changes. The greatest challenge
  for~\cref{rq:3.2} is to explain the \emph{metrics-oriented
  architectural model} to developers, and then test if it improves
  their ability to identify refactoring opportunities.

  \section{Proposal Structure}
  \label{sec:proposal_structure}
  %%%%%%%%%%%%%%%%%%%%%%%%%%%%%%%%%%%%%%%%%%%%%%%%%%%%%%%%%%%%%%%%%%%%%%%%%%%%%%

  The remaining of this proposal is structured as follows.
    \Cref{chap:ml_enabled_systems} formally discusses the definition
    of ML-enabled systems, discussing a reference architecture for
    these systems.
    \Cref{chap:the_spira_system} provides a real-world example of
    an ML-enabled system, designed by the authors of this proposal.
    \Cref{chap:research_methodology} delineates the research
    methodology, whose goal is to answer the research questions
    presented in \cref{chap:introduction}.
    \Cref{chap:work_plan} presents the work plan to execute
    this research.
    Finally, \cref{app:related_activities} lists academic activities
    related to this PhD.
    % \Cref{chap:measuring_complexity} provides a preliminary literature
    % review about software metrics.
