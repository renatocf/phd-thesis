%!TeX root=../tese.tex
%("dica" para o editor de texto: este arquivo é parte de um documento maior)
% para saber mais: https://tex.stackexchange.com/q/78101/183146

% As palavras-chave são obrigatórias, em português e em inglês, e devem ser
% definidas antes do resumo/abstract. Acrescente quantas forem necessárias.
\palavrachave{Complexidade de Software}
\palavrachave{Métricas de Software}
\palavrachave{Sistemas de Aprendizagem de Máquina}
\palavrachave{Engenharia de Aprendizagem de Máquina}
\palavrachave{MLOps}
 

% O resumo é obrigatório, em português e inglês. Estes comandos também
% geram automaticamente a referência para o próprio documento, conforme
% as normas sugeridas da USP.
\resumo{
A crise do software chegou à IA: segundo o relatório da Gartner em 2021, apenas
cerca de 53\% dos produtos baseados em IA chegam à produção. Desde o início da
disciplina de engenharia de software, sabe-se que o aumento da complexidade
realiza um papel de destaque para que projetos falhem. O objetivo desta pesquisa
é investigar como complexidade afeta o Aprendizado de Máquina (em inglês,
\BrToEn{Machine Learning}, ML). Por um lado, sistemas baseados em ML
(em inglês, \BrToEn{ML-enabled systems}) são essencialmente complexos: requerem
múltiplos componentes para treinar e servir modelos de ML, e evoluem ao longo
de três eixos de mudança -- código, modelo, e dados. Por outro lado, lidar com
essa complexidade é essencialmente um combate a incêndio: não há cultura
organizacional, práticas, ou processos bem estabelecidos. Nesse contexto, uma
questão de pesquisa crítica emerge: como gerenciar de maneira efetiva a
complexidade de sistemas baseados em ML? Para responder a essa pergunta, esta
pesquisa visa introduzir um modelo arquitetural baseado em métricas para
caracterizar a complexidade de sistemas baseados em ML. Metodologicamente, as
métricas serviriam para dar suporte às decisões arquiteturais, servindo de guia
para a concepção e evolução desses sistemas. Além disso, a caracterização
ofereceria um ambiente de testes para elaborar estratégias organizacionais para
o futuro. Para entender o estado da arte sobre sistemas baseados em ML, esta
proposta se apoia na literatura da nascente área de Engenharia de Aprendizado de
Máquina (em inglês, \BrToEn{Machine Learning Engineering}, MLE). Embora a
literatura seja rica em métricas de complexidade de software, poucos resultados
exploram como os dados e, em particular, os modelos afetam a complexidade geral
dos sistemas baseados em ML. Consequentemente, esta pesquisa visa liderar essa
discussão. Para criar o modelo de caracterização, ela propõe dois casos de
estudo baseados em sistemas baseados em ML reais. O primeiro será pautado no
projeto SPIRA, desenvolvido na Universidade de São Paulo (USP) no Brasil. O
sistema SPIRA foi arquitetado em 2020 e desenvolvido por múltiplos estudantes de
graduação orientados como parte desta pesquisa. Num estágio inicial, o objetivo
será usar o código e documentação abertos do SPIRA -- incluindo resultados
publicados e em andamento -- para maturar o modelo de caracterização. O segundo
caso de estudo será feito com parceiros da indústria da Academia Jheronimus de
Ciência de Dados (JADS) da Holanda. O JADS tem um ecossistema rico que junta
academia e indústria, e tem hospedado esta pesquisa desde 2023. Num estágio
posterior, o objetivo é validar empiricamente a aplicabilidade e utilizado do
modelo de caracterização em sistemas da indústria. Ao usar métricas efetivas
para acompanhar a complexidade, esperamos que novos sistemas inteligentes tenham
uma probabilidade maior de chegar à produção.
}
